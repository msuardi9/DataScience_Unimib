% Options for packages loaded elsewhere
\PassOptionsToPackage{unicode}{hyperref}
\PassOptionsToPackage{hyphens}{url}
%
\documentclass[
]{article}
\usepackage{amsmath,amssymb}
\usepackage{iftex}
\ifPDFTeX
  \usepackage[T1]{fontenc}
  \usepackage[utf8]{inputenc}
  \usepackage{textcomp} % provide euro and other symbols
\else % if luatex or xetex
  \usepackage{unicode-math} % this also loads fontspec
  \defaultfontfeatures{Scale=MatchLowercase}
  \defaultfontfeatures[\rmfamily]{Ligatures=TeX,Scale=1}
\fi
\usepackage{lmodern}
\ifPDFTeX\else
  % xetex/luatex font selection
\fi
% Use upquote if available, for straight quotes in verbatim environments
\IfFileExists{upquote.sty}{\usepackage{upquote}}{}
\IfFileExists{microtype.sty}{% use microtype if available
  \usepackage[]{microtype}
  \UseMicrotypeSet[protrusion]{basicmath} % disable protrusion for tt fonts
}{}
\makeatletter
\@ifundefined{KOMAClassName}{% if non-KOMA class
  \IfFileExists{parskip.sty}{%
    \usepackage{parskip}
  }{% else
    \setlength{\parindent}{0pt}
    \setlength{\parskip}{6pt plus 2pt minus 1pt}}
}{% if KOMA class
  \KOMAoptions{parskip=half}}
\makeatother
\usepackage{xcolor}
\usepackage[margin=1in]{geometry}
\usepackage{graphicx}
\makeatletter
\def\maxwidth{\ifdim\Gin@nat@width>\linewidth\linewidth\else\Gin@nat@width\fi}
\def\maxheight{\ifdim\Gin@nat@height>\textheight\textheight\else\Gin@nat@height\fi}
\makeatother
% Scale images if necessary, so that they will not overflow the page
% margins by default, and it is still possible to overwrite the defaults
% using explicit options in \includegraphics[width, height, ...]{}
\setkeys{Gin}{width=\maxwidth,height=\maxheight,keepaspectratio}
% Set default figure placement to htbp
\makeatletter
\def\fps@figure{htbp}
\makeatother
\setlength{\emergencystretch}{3em} % prevent overfull lines
\providecommand{\tightlist}{%
  \setlength{\itemsep}{0pt}\setlength{\parskip}{0pt}}
\setcounter{secnumdepth}{-\maxdimen} % remove section numbering
\ifLuaTeX
  \usepackage{selnolig}  % disable illegal ligatures
\fi
\usepackage{bookmark}
\IfFileExists{xurl.sty}{\usepackage{xurl}}{} % add URL line breaks if available
\urlstyle{same}
\hypersetup{
  pdftitle={Decision models - First assignment},
  pdfauthor={Matteo Suardi},
  hidelinks,
  pdfcreator={LaTeX via pandoc}}

\title{Decision models - First assignment}
\author{Matteo Suardi}
\date{2025-04-09}

\begin{document}
\maketitle

\section{Exercise 1}\label{exercise-1}

\#\#Answer the following questions:

\#\#\#\#1. Can an LP model have more than one optimal solution? Is it
possible for an LP model to have exactly two optimal solutions? Why or
why not?

Yes, an LP problem can have more than one optimal solution. This could
happen when the objective function is parallel to one of the problem
constraints that defines on the edge (or face, in higher dimensions) of
the feasible region. In this case, the entire segment of feasible points
along that edge yields the same optimal value and might be an optimal
solution. However, an LP problem can NOT have two exact optimal
solutions. This is because the set of optimal solutions (feasible
region) in linear programming is always convex. If two points \(x_1\)
and \(x_2\) are both optimal, then every convex combination of these two
points (that is, any point of the form
\(\lambda x_1 + (1- \lambda)x_2\), where \(0 \le \lambda \le 1\)) is
also optimal. Therefore, between any two optimal solutions, there are
infinitely many other optimal solutions.

\#\#\#\#2. Are the following objective functions for an LP model
equivalent?

\[\text{max} (2x_1 + 3x_2 −x_3) \\ \text{min}(−2x_1 −3x_2 +x_3)\]

Yes, the two objective functions are equivalent, because maximizing a
linear function is equivalent to minimizing its negative. If both
objective functions are applied to the same set of constraints, they
will lead to the same optimal values for the decision variables
\(x_1,x_2,x_3\). The only difference will be the objective function
value, which will have opposite signs. So yes, both formulations will
produce identical optimal solutions, even though the objective values
will be numerically opposite.

\paragraph{Which of the following constraints are not linear or cannot
be included as a constraint in a linear programming
problem?}\label{which-of-the-following-constraints-are-not-linear-or-cannot-be-included-as-a-constraint-in-a-linear-programming-problem}

The constraints that are not linear and therefore cannot be included in
an LP problem are:

\begin{itemize}
\tightlist
\item
  \(2x_1 + \sqrt x_2 \ge 60\) because it contains a square root, which
  is a non-linear function.
\item
  \(\frac{3x_1+2x_2x_1-3x_3}{x_1+x_2+x_3} \le 0.9\) because it contains
  a product of variables and at least a variable in the denominator,
  both of which make it non-linear.
\item
  \(3x_1^2+7x_2 \le 45\) because it contains a quadratic term \(x1^2\),
  which is not linear.
\end{itemize}

Only the constraints \(2x_1+x_2-3x_3 \ge 50\) and
\(4x_1- \frac{1}{2}x_2=75\).

\section{Exercise 2}\label{exercise-2}

A construction materials company is looking for a way to maximize profit
for transportation of their goods. The company has a train available
with 4 wagons. The weight and surface capacities of each wagon are shown
in the following table:

\begin{figure}
\centering
\includegraphics{table_ex2.png}
\caption{Exercise 2 table}
\end{figure}

\end{document}
